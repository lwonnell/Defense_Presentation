%%%%%%%%%%%%%%%%%%%%%%%%%%%%%%%%%%%%%%%%%
% Beamer Presentation
% LaTeX Template
% Version 1.0 (10/11/12)
%
% This template has been downloaded from:
% http://www.LaTeXTemplates.com
%
% License:
% CC BY-NC-SA 3.0 (http://creativecommons.org/licenses/by-nc-sa/3.0/)
%
%%%%%%%%%%%%%%%%%%%%%%%%%%%%%%%%%%%%%%%%%

%----------------------------------------------------------------------------------------
%	PACKAGES AND THEMES
%----------------------------------------------------------------------------------------

\documentclass{beamer}

\mode<presentation> 

% The Beamer class comes with a number of default slide themes
% which change the colors and layouts of slides. Below this is a list
% of all the themes, uncomment each in turn to see what they look like.

%\usetheme{default}
%\usetheme{AnnArbor}
%\usetheme{Antibes}
%\usetheme{Bergen}
%\usetheme{Berkeley}
%\usetheme{Berlin}
%\usetheme{Boadilla}
%\usetheme{CambridgeUS}
%\usetheme{Copenhagen}
%\usetheme{Darmstadt}
%\usetheme{Dresden}
%\usetheme{Frankfurt}
%\usetheme{Goettingen}
%\usetheme{Hannover}
%\usetheme{Ilmenau}
%\usetheme{JuanLesPins}
%\usetheme{Luebeck}
\usetheme{Madrid}
%\usetheme{Malmoe}
%\usetheme{Marburg}
%\usetheme{Montpellier}
%\usetheme{PaloAlto}
%\usetheme{Pittsburgh}
%\usetheme{Rochester}
%\usetheme{Singapore}
%\usetheme{Szeged}
%\usetheme{Warsaw}

% As well as themes, the Beamer class has a number of color themes
% for any slide theme. Uncomment each of these in turn to see how it
% changes the colors of your current slide theme.

%\usecolortheme{albatross}
%\usecolortheme{beaver}
%\usecolortheme{beetle}
%\usecolortheme{crane}
%\usecolortheme{dolphin}
%\usecolortheme{dove}
%\usecolortheme{fly}
%\usecolortheme{lily}
%\usecolortheme{orchid}
%\usecolortheme{rose}
%\usecolortheme{seagull}
%\usecolortheme{seahorse}
%\usecolortheme{whale}
%\usecolortheme{wolverine}

%\setbeamertemplate{footline} % To remove the footer line in all slides uncomment this line
%\setbeamertemplate{footline}[page number] % To replace the footer line in all slides with a simple slide count uncomment this line

%\setbeamertemplate{navigation symbols}{} % To remove the navigation symbols from the bottom of all slides uncomment this line


\usepackage{graphicx} % Allows including images
\usepackage{booktabs} % Allows the use of \toprule, \midrule and \bottomrule in tables
\usepackage{mathtools}
\usepackage{amsmath}
\usepackage{float}
\usepackage[makeroom]{cancel}
\usepackage{bm}
\usepackage[export]{adjustbox}
\usepackage{caption}
\captionsetup[figure]{labelformat=empty}
\usepackage{subcaption}
\usepackage{multicol}
\usepackage{animate}
\definecolor{ao(english)}{rgb}{0.0, 0.5, 0.0}


%----------------------------------------------------------------------------------------
%	TITLE PAGE
%----------------------------------------------------------------------------------------

\title[Kinetic Analysis of MCT]{A kinetic analysis of morphing continuum theory 
for fluid flows} % %The 
%short title appears at the bottom of every slide, the full title is only on 
%the title page

\author[Wonnell]{\Large Louis Blais Wonnell} % Your name
\vspace{50mm}
\institute[K-State] % Your institution as it will appear on the bottom of slide,
{\large Major Professor: Dr. James Chen \\ 
\medskip
\normalsize Department of Mechanical and Nuclear 
Engineering, \\ Kansas State University \\ % Your institution for the title page
\medskip
%\textit{lwonnell@ksu.edu} % Your email address
}
\date[PhD Defense]
{\large April 20th, 
2018} % Date, can be %changed to a custom date
\begin{document}
\setbeamertemplate{caption}{\raggedright\insertcaption\par}
\begin{frame}
\titlepage % Print the title page as the first slide
\end{frame}

% \begin{frame}
% \frametitle{Compressible Flow Simulations} % Table of contents slide, comment this block out to remove it
% To test the effectiveness of numerical MCT simulations on compressible flow, the 
% case of uniform flow over a stationary cylinder will be investigated. The intent 
% is to simulate the vortex shedding frequency, through the Strouhal number, as a 
% function of the Reynolds number, and compare with experimental data. For this 
% verification process, experimental data will come from Roshko's experiments on a 
% circular cylinder using the Southern California Co-operative Wind Tunnel 
% (CWT)\footnote{Roshko, 1960).\end{frame}
\begin{frame}
\frametitle{Agenda}
\begin{multicols}{2}
\begin{figure}
 \includegraphics[width=.57\linewidth]{LiquidCrystals.jpg}
 \caption{\tiny (PeerScientist, 2017)}
\end{figure}
 \begin{figure}
\includegraphics[width=.6\linewidth]{kineticgases.jpg}
\caption{\tiny (myChemset.com)}
\end{figure}
\end{multicols}
\begin{multicols}{2}
  \centering
  \begin{itemize}
  \small
  \item Motivation: Efficient Modeling of Flows with Local Rotation 
   \item Previous Efforts to Improve Direct Numerical Simulation (DNS)
   \item A Multi-scale Approach: Morphing Continuum Theory (MCT)
   \item Kinetic Theory and the Physics of MCT
   \item Assessing the Equations: Numerical Results of MCT
   \item Discussion
   \item Conclusion and Future Work
  \end{itemize}
\end{multicols}
\end{frame}


%----------------------------------------------------------------------------------------
%	PRESENTATION SLIDES
%----------------------------------------------------------------------------------------

%------------------------------------------------
 % Sections can be created in order to organize your 
%presentation into discrete blocks, all sections and subsections are 
%automatically printed in the table of contents as an overview of the talk
%------------------------------------------------

%\subsection{Subsection Example} % A subsection can be created just before a set 
%of slides with a common theme to further break down your presentation into

\begin{frame}
 \frametitle{Motivation: Efficient Modeling of Flows with Local Rotation}
 \begin{multicols}{2}
  \begin{figure}
   \includegraphics[width=\linewidth]{Hypersonic_Flow_Ivey_2011.png}
%    \caption{(Ivey, 2011)}
  \caption{\tiny (Ivey et al, 2011)}
  \end{figure}
  \begin{figure}
   \includegraphics[width=\linewidth]{NASAflowcyl.jpg}
%    \caption{(NASA, 2015)}
   \caption{\tiny (NASA, 2015)}
  \end{figure}
 \end{multicols}
\end{frame}

\begin{frame}
\frametitle{Previous Approaches to Improve DNS}
\textbf{\Large Direct Numerical Simulation (DNS)}
\begin{multicols}{2}
\begin{itemize}
\item \textbf{Constraint 1}: Number of elements in mesh, $N$, and mesh element 
size $h$, must satisfy $Nh > L$, for integral length scale $L$ (i.e. box length)
\pause
\item \textbf{Constraint 2}: Mesh element size $h$ must be smaller than 
smallest physical length scale $\eta$
\pause
\item \textbf{Computational cost} rises when $\eta$ is small and $L$ is large
\pause
\item \textit{Atmospheric Turbulence}: $\eta \sim 0.1mm$ while $L \sim
10 km$ $\rightarrow$ \textbf{Large range of eddy scales}
\end{itemize}
\begin{figure}
 \includegraphics[width=.55\linewidth]{Atmospheric-Turbulence.jpg}
 \caption{\tiny (Shats, 2012) }
\end{figure}
\end{multicols}
\end{frame}
\begin{frame}
\frametitle{Previous Approaches to Improve DNS}
\textbf{\Large Direct Numerical Simulation}
\begin{multicols}{2}
\begin{itemize}
\item \textit{Vortex Flow}: $\eta$ might represent length of vortex core, 
$\eta$ not clear due to problem of identifying vortices
\item Real flows are \textbf{multi-scale}, requiring flexibility in numerical approach
\end{itemize}
\begin{figure}
 \includegraphics[width=\linewidth]{vortexaneurysm.jpg}
 \caption{\tiny (Elster, 2017)}
\end{figure}
\end{multicols}
\end{frame}
\begin{frame}
 \frametitle{Previous Approaches to Improve DNS}
 \begin{table}[t!]
 \begin{tabular}{|| c | c | c ||}
  \hline
  Method & Approach & Contribution \\ [0.5ex]
  \hline\hline
  RANS & \scriptsize Average N-S eq's & \scriptsize Distinguishes mean flow from fluctuation\\
  \hline
  LES & \scriptsize Filter N-S eq's & \scriptsize Reduces cost associated with subgrid modeling\\
  \hline
  DES & \scriptsize Combines RANS and LES & \scriptsize Keeps more physics at subgrid scales where relevant \\
  \hline
 \end{tabular}
 \end{table}
  \begin{figure}
  \includegraphics[width=.375\linewidth]{MeshScales.png}
 \end{figure}
\end{frame}
\begin{frame}
\frametitle{Previous Approaches to Improve DNS}
 \begin{figure}
  \includegraphics[width=.9\linewidth]{ModelPyramid.jpg}
 \end{figure}
 %\end{multicols}
\end{frame}
\begin{frame}
 \frametitle{Previous Approaches to Improve DNS}
 \Large
 Drawbacks
 \pause
 \begin{itemize}
  \item Limits range of parameters for simulation
  \item Require \textbf{closure models}
  \item Nature of closure models may be \textit{ad-hoc}
  \item Models modified when new data emerges
 \end{itemize}
 \vspace{5mm}
 \pause
 Desire for a \textbf{multiscale} theory that eases burden of mesh refinement
\end{frame}

\begin{frame}
\Large
Motion and deformation of particles built into fluid picture
\frametitle{A Multi-scale Approach: Morphing Continuum Theory (MCT)}
\begin{multicols}{2}
\begin{figure}
\includegraphics[width=1.1\linewidth]{CoordinatesMCT.jpg}
\end{figure}
\large
\textbf{Vector Relations}
\begin{equation*} %\label{direc1}
%\begin{split}
\centering
{\it x_{k} = x_{k} (X_{K},t)}
\quad
{\it  X_{K} = X_{K} (x_{k},t)}
\end{equation*}
\begin{equation*} %\label{direc2}
\centering
{\it \xi_{k} = \chi_{kK}(X_{K},t)\Xi_{K}}
\quad
{\it \Xi_{K} =  \bar{\chi}_{Kk}\xi_{k}}
\end{equation*}
\begin{equation*} %\label{direc3}
\centering
{\it K = 1, 2, 3} 
\quad
{\it  k = 1, 2, 3 }
\end{equation*}
\textbf{Deformation Rates}
\begin{equation*}
 a_{kl}  = v_{l,k} + \epsilon_{lkm}\omega_{m}
\end{equation*}
\begin{equation*}
 b_{kl} = \omega_{k,l}
\end{equation*}
\end{multicols}
\end{frame}

\begin{frame}
\frametitle{A Multi-scale Approach: Morphing Continuum Theory (MCT)}
\centering
\begin{multicols}{2}
%\begin{equation}
\textbf{Continuity}
$$\frac{\partial \rho}{\partial t} + (\rho v_{i})_{,i} = 0$$\\
%\end{equation}
%\begin{equation}
\textbf{Linear Momentum}
$$t_{lk,l} + \rho f_{k} =  \rho {\dot{v}}_{ k} $$\\
%\end{equation}
%\begin{equation}
%\end{equation} 
\textbf{Angular Momentum}
$$ m_{lk,l} + \epsilon_{kij}t_{ij} + \rho l_{k} = \rho j\dot{\omega}_{ k}$$\\
%\begin{equation}
\textbf{Energy}
$$\rho \dot{e} - t_{kl}a_{kl} - m_{kl}b_{lk} + q_{k,k} - \rho h = 0$$\\
\end{multicols}
\begin{multicols}{2}
\begin{itemize}
\item $t_{lk}$: Cauchy stress 
\item $m_{lk}$: Moment stress 
\item $j$: Microinertia of particles
\item $a_{bkl}, b_{kl}$: Defomation Rates
\item $q_{k}$: Heat flux
\item $h$: Energy source density
\item $e$: Internal energy density
\end{itemize}
\end{multicols}
%\end{equation} 
\end{frame}

\begin{frame}
\frametitle{A Multi-scale Approach: Morphing Continuum Theory (MCT)}
\small
Constitutive equations derived from Clausius-Duhem Inequality for isotropic 
fluids. Stress tensors must be frame-indifferent or \textbf{objective}.
\begin{multicols}{2}
\centering
\textbf{\large Cauchy Stress}
\normalsize
$$t_{kl} = -P\delta_{kl} + \lambda tr(a_{mn})\delta_{kl} + (\mu + {\color{blue}\kappa})a_{kl} + \mu a_{lk} $$\\
\textbf{\large Moment Stress}
\normalsize
{\color{violet}$$m_{kl} = \frac{\alpha_{T}}{\theta} \epsilon_{klm} \theta_{,m} + \alpha tr(b_{mn})\delta_{kl} + \beta b_{kl} + \gamma b_{lk} $$}\\
\textbf{\large Heat Flux}
\normalsize
$$q_{k} = \frac{K}{\theta}\theta_{,k} + 
{\color{red}\alpha_{T}\epsilon_{klm}\omega_{m,l} } $$\\
\textbf{\large Deformation Rates}
$$ a_{kl}  = v_{l,k} + {\color{magenta}\epsilon_{lkm}\omega_{m}}$$
{\color{violet}$$ b_{kl} = \omega_{k,l} $$}
\end{multicols}
\end{frame}
\begin{frame}
\frametitle{A Multi-scale Approach: Morphing Continuum Theory (MCT)}
\small
Now have independent equation governing \textbf{gyration} of 
particles. The coefficient $\color{blue} \kappa$ ties particle rotation to 
linear momentum equation. 
\vfill
\centering
\textbf{Linear Momentum}
\normalsize
$$-P_{,k} + (\lambda + \mu)v_{l,lk} + (\mu + {\color{blue}\kappa})v_{k,ll} + {\color{blue}\kappa\epsilon_{klm}\omega_{l,m}} = \rho\dot{v}_{k}$$ \\
\textbf{\color{violet}Angular Momentum}
\normalsize
$$\it (\alpha + \beta)\omega_{l,lk} + \gamma \omega_{k,ll} + 
{\color{blue}\kappa(\overbrace{\epsilon_{klm}v_{l,m} - 
2\omega_{k})}^{{\color{black}Absolute \ Rotation}}} = \rho j\dot{\omega}_{k}  
$$\\
\pause
\textbf{Energy}
$$({\color{blue}t_{kl}}v_{l})_{,k} + {\color{violet}(m_{kl}\omega_{l})_{,k}} - q_{k,k} + \rho h  = \rho{\color{cyan}\dot{E}} $$ \\
\textnormal
where $E = e + \frac{1}{2}(v_{l}v_{l} + {\color{cyan}j\omega_{l}\omega_{l}})$ 
now contains rotational kinetic energy. 
\end{frame}

\begin{frame}
 \frametitle {Kinetic Theory and the Physics of MCT}
 \Large
 New coefficients in MCT are \textbf{mathematical consequences} from C-D 
inequality and Axiom of Objectivity \pause $\rightarrow$ No inherent physical 
meaning \\
\pause
\vspace{5mm}
In classical theory, kinematic viscosity, $\mu$, often depends on 
\textbf{temperature} \pause $\rightarrow$ Kinetic Theory shows this dependence 
(Huang, 1987) \\
\vspace{5mm}
\pause
To give new terms in MCT a deeper physical meaning \pause $\rightarrow$ Kinetic 
Theory for gases composed of rotating particles
\end{frame}


\begin{frame}
\frametitle{Kinetic Theory and the Physics of MCT}
\large
\textbf{Basic Approach of Kinetic Theory}
\begin{itemize}
\item Assess the behavior of a fluid by modeling an aggregation of particles 
(atoms or molecules)
\item Precise position or motion of any particle is not determined
\item When the number of particles is large enough $\rightarrow$ 
\textbf{Probability Distribution} models collective behavior of particles
%\item Important to define \textbf{phase space}
\item Distribution reformulated depending on nature of particles
\item Bottom-up approach to obtaining governing equations for fluid motion 
\end{itemize}
\pause
\end{frame}

\begin{frame}

\frametitle{Kinetic Theory and the Physics of MCT}

\textbf{\large Maxwell-Boltzmann Theory}
\begin{multicols}{2}

\begin{figure}
 \includegraphics[width=45mm]{Boltzmann.png}
\end{figure}
\small
\centering
(Bhardwaj, 2014) 
\begin{itemize}
 \item Focused on case of monatomic gases or point particles
 \item \textbf{Mean free path} small compared to characteristic length of 
system 
$\rightarrow$ Reasonably large number of collisions
 \item Variables of interest: $x_{i}$, $v_{i}$, $t$
 \item Mean velocity $U_{i}$, mean thermal energy $\theta$ and number density 
$n$, functions of $x_{i}$ and $t$
\item Boltzmann Transport Equation \pause $\rightarrow$ 
Euler/Navier-Stokes Equations
\pause
\item Physics behind fluid properties illuminated by kinetic theory
\end{itemize}
\end{multicols}
\Large
%  \begin{equation*}
% \color{red} f(\vec{r}, \vec{v}, t) = \frac{n}{(2\pi 
% m\theta)^{3/2}}exp(-\frac{m\vert\vec{v} 
% - \vec{U}\vert^2}{2\theta})
% \end{equation*}
\end{frame}

\begin{frame}
\frametitle{Kinetic Theory and the Physics of MCT}
 \textbf{\large Fluids with Internal Structure}
 \begin{itemize}
  \item New variable of interest: $\omega_{i}$ $\rightarrow$ Gyration of inner 
structures
  \item Total linear momentum includes this gyration: $p_{i} = m(v_{i} + 
\epsilon_{ijk}r_{j}\omega_{k}) $
  \item Chen accounted for gyration in Boltzmann-Curtiss distribution 
(Chen, 2017) $\rightarrow$ Zeroth-order equations matched inviscid 
\textbf{Morphing Continuum Theory} (MCT) equations 
\pause
\item Boltzmann-Curtiss transport equation
 \end{itemize}
\small
  \begin{equation*}
  (\frac{\partial}{\partial t} + v_{i} \frac{\partial}{\partial x_{i}} + 
\frac{M_i}{I}\frac{\partial}{\partial \Phi_i})f = 
(\frac{\partial f}{\partial t})_{coll}
 \end{equation*}
 Here, $\Phi_i$ is the orientation of a particle, $M_i$ is the angular 
momentum, and $I$ is the moment inertia. With spherical particles, 
$\frac{\partial}{\partial \Phi_i} = 0$, yielding:
  \begin{equation*}
  (\frac{\partial}{\partial t} + v_{i} \frac{\partial}{\partial x_{i}})f = 
(\frac{\partial f}{\partial t})_{coll}
 \end{equation*}
\end{frame}
\begin{frame}
 \frametitle{Kinetic Theory and the Physics of MCT}
 \begin{multicols}{2}
 \begin{figure}
 \includegraphics[width=.85\linewidth]{PhaseSpace.png}
\end{figure}
 Additional rotation changes dimensionality of the phase space \\
 \vspace{5mm}
Number density $$n = \int \int d^3v' d^3\omega' \ f$$ 
Any integral of a quantity in this phase space will include effects of particle 
rotation
\end{multicols}
\end{frame}
\begin{frame}
 \frametitle{Kinetic Theory and the Physics of MCT}
 \textbf{Boltzmann-Curtiss Distribution}: Derived from Boltzmann's Principle $S 
= kln(W)$ for system of rotating particles (Chen, 2017)
\begin{equation*}
%\label{BoltzmannCurtissSecond}
f^{0}(x_{i}, v_{i}, \omega_{i}, t) = n(\frac{m\sqrt{j}}{2\pi 
\theta})^3 exp(-\frac{m(v'_{j}v'_{j} + j\omega'_{p}\omega'_{p})}{2\theta})
\end{equation*}
\small
\begin{multicols}{2}
\begin{itemize}
\item $S$: Total entropy of the system
\item $k$: Boltzmann's constant
\item $W$: Number of microstates consistent with given macrostate
\item $m$: Mass of a particle
\item $n$: Number density of particles
\item $j$: Microinertia of particles
\item $\theta$: Equilibrium thermal energy, $kT$ 
\item $v'_k$: Fluctuation in translational velocity
\item $\omega'_{k}$: Fluctuation in particle rotation
\end{itemize}
\end{multicols}
\end{frame}


\begin{frame}
 \frametitle{Kinetic Theory and the Physics of MCT}
 \textbf{\large Collisional Integral}
  \begin{align*}
 \label{collisionintegral}
 (\frac{\partial f}{\partial t})_{coll} =  \int &d^3 v_2 \ d^3\omega_2 \ 
d^3v_1' \ d^3\omega_1' \ d^3v_2' \ d^3\omega_2' \ \delta^4(P_f - 
P_i)\left|T_{fi}\right|^2 \\ &(f_2'f_1' - f_2 f_1)
\end{align*}
 \begin{multicols}{2}
 \begin{figure}
  \includegraphics[width=\linewidth]{BinaryCollision.png}
 \end{figure}
\begin{itemize}
\small
\item $P_{f}$: Total Final Momentum  
\item $P_{i}$: Total Initial Momentum
\item $T_{fi}$: Transition Matrix
\item Particles 1 and 2 collide in binary collision
\item Unprimed quantities correspond to initial conditions
\item Primed quantities correspond to final conditions
\end{itemize}
\end{multicols}
\end{frame}

\begin{frame}
 \frametitle{Kinetic Theory and the Physics of MCT}
 \textbf{\large Conservation Equation}
 Let $\chi(x_i, p_i)$ be a conserved quantity in a binary collision whose total 
value is preserved before and after the collision, i.e.:
\begin{equation*}
 \chi_1 + \chi_2 = \chi_1' + \chi_2'
\end{equation*}
The Boltzmann Equation can be integrated with respect to this conserved 
property, yielding:
\begin{equation*}
  \frac{\partial }{\partial t} \langle n\chi
\rangle + \frac{\partial}{\partial x_{i}} \langle n
\frac{p_{i}}{m}\chi
\rangle - n \langle \frac{p_{i}}{m}\frac{\partial \chi}{\partial x_{i}}
\rangle = 0
\end{equation*}
where
\begin{equation*}
  \langle A \rangle = \frac{1}{n} \int \int A f(x_{i}, v_{i}, \omega_{i}, t) 
d^3v' d^3\omega'
\end{equation*}
The RHS, $\langle \frac{\partial f}{\partial t}_{coll} \chi \rangle $, vanishes 
due to the symmetry, (Huang, 1987) $$\langle \textbf{p}'_2, \textbf{p}'_1 
|T|\textbf{p}_1, \textbf{p}_2\rangle = \langle -\textbf{p}'_2, 
-\textbf{p}'_1 |T| -\textbf{p}_1,-\textbf{p}_2\rangle$$
\end{frame}

\begin{frame}
 \frametitle{Kinetic Theory and the Physics of MCT}
 \textbf{\large Conservation Laws}\\
 \vspace{2mm}
 Obtained by letting $\chi$ equal any quantity conserved in the binary 
collision, i.e. $\chi_1 + \chi_2 = \chi_1' + \chi_2'$ \\
 \begin{align*}
 \centering
 \small
\label{firstordercont}
&\textbf{Continuity} \ \Big(\chi_1 = m\Big) \nonumber \\
&\frac{\partial }{\partial t}\langle mn
\rangle + \frac{\partial }{\partial x_i}\langle mnv_i
\rangle = 0 \\
\label{firstordermoment}
&\textbf{Linear Momentum} \ \Big(\chi_2 = m(v_i + 
\epsilon_{ipl}r_l\omega_p)\Big) \nonumber \\
&\frac{\partial}{\partial t}\langle mn v_i
\rangle + \frac{\partial }{\partial t}\langle mn 
\epsilon_{ipl}r_{l}\omega_{p}
\rangle + \frac{\partial}{\partial x_{l}}\langle mn v_iv_l
\rangle + 
\\ &\frac{\partial }{\partial x_s}\langle mn\epsilon_{ipl} v_s r_{l}\omega_{p}
\rangle = 0 
\nonumber \\
\end{align*}
\end{frame}

\begin{frame}
 \frametitle{Kinetic Theory and the Physics of MCT}
 \textbf{\large Conservation Laws}\\
\begin{align*}
\centering
\label{firstorderangmoment}
&\textbf{Angular Momentum} \ \Big(\chi_3 = mr_ir_p\omega_p \Big) \nonumber \\
&\frac{\partial}{\partial t}\langle mn r_{i}r_{p}\omega_p
\rangle + 
\frac{\partial}{\partial x_{l}}\langle mn r_{i}r_{p}\omega_pv_l
\rangle = 0 \\
\label{firstordereenergy}
&\textbf{Energy} \ \Big(\chi_4 = m(e + \frac{1}{2}[v'_lv'_l + 
r_pr_q\omega'_p\omega'_q])\Big) \nonumber \\
&\frac{\partial}{\partial t}(mn e) + \frac{\partial}{\partial 
x_{i}}(mnev_i)   + \frac{\partial}{\partial x_{i}}\frac{1}{2}\langle mn 
v'_{l}v'_{l}v'_{i} + r_{p}r_{q}\omega'_{p}\omega'_{q}v'_{i}
\rangle - \\  &  
mn\langle v_{i}\frac{\partial e}{\partial x_{i}}
\rangle = 0 \nonumber
 \end{align*}
\end{frame}

\begin{frame}
 \frametitle{Kinetic Theory and the Physics of MCT}
 Applying the principles $\langle v'\rangle = \langle \omega'\rangle = 0$ and 
$\langle \epsilon_{ipl} v_s r_{l}W_{p} \rangle = 0$, assuming no external 
torque on the system (Baraff, 1997). Letting $i_{pq} = r_pr_q$.
  \begin{align*}
\label{firstordercont2}
&\textbf{Continuity} \nonumber \\
&\frac{\partial }{\partial t}\rho + \frac{\partial }{\partial 
x_l}(\rho U_l) = 0 \\
\label{firstordermoment2}
&\textbf{Linear Momentum} \nonumber \\ 
&\frac{\partial}{\partial t}(\rho U_s)  + \frac{\partial}{\partial x_l}(\rho 
U_sU_l) + \frac{\partial }{\partial x_l}(\rho \langle v'_s v'_l
\rangle + \langle \rho 
\epsilon_{spq} v'_l r_{q}\omega'_{p}
\rangle) = 0\\
\label{firstorderangmoment2}
&\textbf{Angular Momentum} \nonumber \\ 
&\frac{\partial}{\partial t}(\rho i_{sp}W_p) + 
\frac{\partial}{\partial x_{l}}(\rho i_{sp}W_pU_l) + \frac{\partial}{\partial 
x_{l}}(\rho \langle i_{sp}\omega'_p v'_l
\rangle) = 0 \\ \nonumber 
\label{firstordereenergy2}
&\textbf{Energy} \nonumber \\ 
&\frac{\partial}{\partial t}(\rho e) + \frac{\partial}{\partial 
x_{l}}(\rho eU_l)   + \frac{\partial}{\partial x_{l}}\frac{1}{2}\langle \rho 
v'_{s}v'_{s}v'_{l} + i_{pq}\omega'_{q}\omega'_{p}v'_{l}
\rangle  -  
\rho \langle v_{l}\frac{\partial e}{\partial x_{l}}
\rangle = 0 
\end{align*}
\end{frame}

\begin{frame}
 \frametitle{Kinetic Theory and the Physics of MCT}
For spherical particles $i_{pq} = i_{pp} = \frac{3j}{2}$, where $j$ is termed 
the microinertia (Chen, 2017).
  \begin{align*}
\label{firstordercont3}
&\textbf{Continuity} \nonumber \\
&\frac{\partial }{\partial t}\rho + \frac{\partial }{\partial 
x_l}(\rho U_l) = 0 \\
\label{firstordermoment3}
&\textbf{Linear Momentum} \nonumber \\ 
&\frac{\partial}{\partial t}(\rho U_s)  + \frac{\partial}{\partial x_l}(\rho 
U_sU_l) + \frac{\partial }{\partial x_l}(\rho \langle v'_s v'_l
\rangle + \langle \rho 
\epsilon_{spq} v'_l r_{q}\omega'_{p}
\rangle) = 0\\
\label{firstorderangmoment3}
&\textbf{Angular Momentum} \nonumber \\ 
&\frac{\partial}{\partial t}(\frac{3\rho jW_s}{2}) + 
\frac{\partial}{\partial x_{l}}(\frac{3\rho jW_sU_l}{2}) + 
\frac{\partial}{\partial 
x_{l}}\rho\langle\frac{3 j\omega'_s v'_l}{2}
\rangle = 0 \\ \nonumber 
\label{firstordereenergy3}
&\textbf{Energy} \nonumber \\ 
&\frac{\partial}{\partial t}(\rho e) + \frac{\partial}{\partial 
x_{l}}(\rho eU_l)   + \frac{\partial}{\partial x_{l}}\frac{1}{2}\langle \rho 
v'_{s}v'_{s}v'_{l} + \frac{3j\omega'_{p}\omega'_{p}v'_{l}}{2}
\rangle  -  
\rho \langle v_{l}\frac{\partial e}{\partial x_{l}}
\rangle = 0 
\end{align*}
\end{frame}

\begin{frame}
 \frametitle{Kinetic Theory and the Physics of MCT}
 Conservation equations give formulations for key forces derived by integration:
 \begin{align*}
   &\textbf{Heat} \ \textbf{Flux}\\
q_{\alpha} &= \frac{1}{2}\langle \rho v'_{l}v'_{l}v'_{\alpha} + 
\frac{3j\omega'_{p}\omega'_{p}v'_{\alpha}}{2}
\rangle \\
&\textbf{Boltzmann} \ \textbf{Stress}\\
t^\text{Bol}_{\alpha \beta} &= -\rho \langle v'_\alpha v'_\beta
\rangle \\
&\textbf{Curtiss} \ \textbf{Stress}\\
t^\text{Cur}_{\alpha \beta} &= - \rho \langle v'_\alpha 
\epsilon_{\beta pq}r_{q}\omega'_{p}
\rangle \\
&\textbf{Moment} \ \textbf{Stress}\\
m_{\alpha \beta} &= -\rho \langle \frac{3j\omega'_\beta v'_\alpha}{2}
\rangle
 \end{align*}
\end{frame}

\begin{frame}
\frametitle{Kinetic Theory and the Physics of MCT}
Evaluation of Stress and Heat Flux Tensors: Zeroth Order (Chen, 2017)
\small
\begin{align*}
\begin{split}
 \label{zeroheatfluxkinetic}
q^{0}_{\alpha} &= \frac{m\rho}{2n}\int (v_l'v_l'v_\alpha' + 
j\omega'_p\omega'_pv_\alpha') (\frac{m\sqrt{j}}{2\pi\theta})^3 
 exp(-\frac{m(v'^2 + j\omega'^2)}{2\theta})\ 
d^3v'd^3\omega' \\ &= \mathbf{0}
\end{split}
\\
\begin{split}
 \label{zeroBoltzmannstress}
t^{\text{Bol}, 0}_{\alpha \beta} &= -\rho\int 
v_\alpha'v_\beta' 
(\frac{m\sqrt{j}}{2\pi\theta})^3 
exp(-\frac{m(v'^2 + j\omega'^2)}{2\theta}) d^3v' d^3 
\omega' \\ 
&= \mathbf{-n}\bm{\theta\delta_{\alpha \beta}}
\end{split}
\\
\begin{split}
 \label{zeroCurtissstress} 
t^{\text{Cur}, 0}_{\alpha \beta} &= -\rho\epsilon_{\beta 
pq}r_{p}\int 
\omega_q'v_{\alpha}' 
(\frac{m\sqrt{j}}{2\pi\theta})^3 exp(-\frac{m(v'^2 + 
j\omega'^2)}{2\theta}) d^3v' d^3 \omega'\\ 
&= \mathbf{0} 
\end{split}
\\
\begin{split}
 \label{zeromomentstress}
 m^0_{\alpha \beta} &= -\frac{3\rho j}{2} \int 
\omega_{\beta}'v_{\alpha}' (\frac{m\sqrt{j}}{2\pi\theta})^3 
exp(-\frac{m(v'^2 + j\omega'^2)}{2\theta}) d^3v' d^3 \omega' 
\\ &= \mathbf{0} 
\end{split}
\end{align*}
\pause
$\rightarrow$ \textbf{Note}: Terms of the form, $\langle v`^{n}\omega'^{m}f^{0} 
\rangle = 0$ when either $n$ or $m$ is odd.
\end{frame}

\begin{frame}
 \frametitle{Kinetic Theory and the Physics of MCT}
Letting $g(x_i, v_i, \omega_i, t) = f(x_i, v_i, \omega_i, t) - f^{0}(x_i, v_i, 
\omega_i, t)$:
 \begin{equation*}
 \begin{split}
(\frac{\partial f}{\partial t})_{coll} = &\int d^3 v_2 \ d^3\omega_2 \ d^3v_1' 
\ d^3 \omega_1' \ d^3v_2' \ d^3\omega_2' \ \delta^4(P_f - 
P_i)\left|T_{fi}\right|^2\\  &[(f'^{0}_2 + g'_2)(f'^{0}_1 + g'_1) - (f^{0}_2 + 
g_2)(f^{0}_1 + g_1)]
\end{split}
 \end{equation*}
Noting that all equilibrium distributions are the same, and excluding quadratic 
terms:
\begin{equation*}
\begin{split}
 (\frac{\partial f}{\partial t})_{coll} \approx &\int d^3 p_2 \ d^3p_1' \ 
d^3p_2' \ \delta^4(P_f - P_i)\left|T_{fi}\right|^2\\  & (f'^{0}_2g'_1 - 
f^{0}_2g_1 + f'^{0}_1g'_2 - g_2 f^{0}_1)
 \end{split}
\end{equation*}
\end{frame}

\begin{frame}
 \frametitle{Kinetic Theory and the Physics of MCT}
 Order of magnitude determined from evaluation of a typical term:
\begin{equation*}
 -g_1 \int d^3 v_2 \ d^3\omega_2 \ d^3 v_1' \ d^3 \omega_1' \ 
d^3v_2' \ d^3\omega_2' \delta^4(P_f - P_i)\left|T_{fi}\right|^2 f_2^{0} = 
-\frac{g}{\tau}
\end{equation*}
 \large 
Therefore the effect of the collisions on the distribution is approximated 
using a single time constant known as \textit{relaxation time}:
\begin{equation*}
(\frac{\partial f}{\partial t})_{coll} \approx - \frac{g}{\tau} 
\end{equation*}
Departures from equilibrium for either of the variables occur over $\tau$ 
before equilibrium is reached \pause $\rightarrow$ \textbf{Experimentally 
determined} \\
\pause
\vspace{2mm}
For $H_2$ mixture, at $p = 1$ atm and $T= 77K$, $\tau_{rot} = 2.20 
\times 10^{-8}s$ (Montero, 2014)
\end{frame}

\begin{frame}
  \frametitle{Kinetic Theory and the Physics of MCT}
  Boltzmann Equation With Collisional Term
  \begin{equation*}
\label{FirstOrderTransport}
 g = -\tau (\frac{\partial}{\partial t} + v_{i} \frac{\partial}{\partial 
x_{i}})(f^{0} + g)
\end{equation*}
Assume $g << f^{0}$
\begin{equation*}
 \label{FirstOrderTransportSimple}
  g = -\tau (\frac{\partial}{\partial t} + v_{i} \frac{\partial}{\partial 
x_{i}})f^{0}
\end{equation*}
Equation contains spatial and temporal derivatives but $f^{0}$ depends on these 
variables implicitly \pause $\rightarrow$ Need derivatives of explicit terms in 
function
\begin{equation*}
%\label{BoltzmannCurtissSecond}
f^{0}(x_{i}, v_{i}, \omega_{i}, t) = n(\frac{m\sqrt{j}}{2\pi 
\theta})^3 exp(-\frac{m(v'_{j}v'_{j} + j\omega'_{p}\omega'_{p})}{2\theta})
\end{equation*}
\end{frame}


\begin{frame}
\frametitle{Kinetic Theory and the Physics of MCT}
 Using chain rule
\begin{equation*}
 g = -\tau(D(\rho)\frac{\partial f^0}{\partial \rho} + D(U_{i})\frac{\partial 
f^0}{\partial U_i} + D(W_i)\frac{\partial f^0}{\partial W_i} + 
D(\theta)\frac{\partial f^{0}}{\partial \theta})
\end{equation*}
where $D(\chi) = (\frac{\partial }{\partial t} + v_i\frac{\partial}{\partial 
x_i})\chi$, is the material derivative. Material derivatives found from zeroth 
order balance laws:\\
\pause
\begin{align*}
\label{zeroordercontMCT}
	\frac{\partial\rho}{\partial t} + \frac{\partial \rho U_l}{\partial 
x_l} = 0 &\rightarrow  D(\rho)=v_l'\frac{\partial}{\partial 
x_l}\rho-\rho\frac{\partial 
U_q}{\partial x_q}\\
\label{zeroordermomentMCT}
	\frac{\partial}{\partial t}(\rho U_s)+  \frac{\partial }{\partial 
x_{l}}(\rho U_lU_s)
=-\frac{\partial }{\partial x_s} (n\theta) &\rightarrow D(U_i)= 
v_l'\frac{\partial}{\partial x_l}U_i-\frac{1}{\rho}\frac{\partial 
}{\partial x_i} (n\theta)\\
\label{zeroorderangmomentMCT}
	\frac{\partial}{\partial t}(\rho jW_s) + \frac{\partial }{\partial 
x_l} (\rho jW_sU_l)
= 0 &\rightarrow D(W_i) = v_l'\frac{\partial}{\partial 
x_l}W_i \\
\label{zeroordereenergyMCT}
	\frac{\partial}{\partial t}(n\theta)+ \frac{\partial }{\partial x_l} 
(n\theta U_l)
=-\frac{n\theta}{3} \frac{\partial U_q}{\partial x_q} &\rightarrow 
D(\theta)=v_l'\frac{\partial}{\partial x_l}\theta-\frac{1}{3}\theta 
\frac{\partial U_q}{\partial x_q}
\end{align*}
\end{frame}

\begin{frame}
 \frametitle{Kinetic Theory and the Physics of MCT}
 \Large \textbf{First-Order Distribution Function}
 \normalsize
 \begin{equation*}
\label{gfull}
\begin{split}
g=-\tau f^{(0)}&[\frac{1}{\rho}(v_i'\frac{\partial \rho}{\partial x_i} -\rho 
\frac{\partial U_i}{\partial x_i}) 
- (\frac{3}{\theta}- 
\frac{m(v'^2+j\omega'^2)}{2\theta^2})(v'_{i}\frac{\partial \theta}{\partial 
x_i} - 
\frac{\theta}{3} \frac{\partial U_q}{\partial x_q}) \\ &+  
(\frac{mv'_i}{\theta})(v_l'\frac{\partial U_i}{\partial x_l} 
-\frac{1}{\rho}\frac{\partial}{\partial x_i}(n\theta)) + 
(\frac{mj\omega_i'}{\theta})(v_l'\frac{\partial W_i}{\partial x_l})]
\end{split}
\end{equation*}
\\ 
\pause
\Large
Now have distribution function need for finding \textbf{first-order} 
approximations to stresses and heat flux
\end{frame}

\begin{frame}
\frametitle{Kinetic Theory and the Physics of MCT}
Evaluation of Stress and Heat Flux Tensors: \textbf{First-Order Heat Flux}
\begin{align*}
 \begin{split}
 \label{heatMCTsurface}
 q^1_{\alpha} = &\frac{m\rho}{2n}\int (v_l'v_l'v_\alpha' + 
j\omega'_p\omega'_pv_\alpha') \ g \ 
d^3v'd^3\omega'  \\
&= \bm{-(4n\tau\theta)\frac{\partial \theta}{\partial x_\alpha}} 
\end{split}
\end{align*}
\pause
\textbf{Note 1}: Heat flux fits form of \textbf{Fourier's Law}:
\begin{equation*}
 q_{\alpha} = -K\frac{\partial \theta}{\partial x_{\alpha}}
\end{equation*}
\pause
\textbf{Note 2}: Volumetric integrals are presumed to have \textit{spherical 
symmetry} i.e. $$\int \int d^3 v' \ d^3\omega' = (4\pi)^2 \int \int v'^2 \ 
\omega '^2 \ dv' \ d\omega'$$
\end{frame}

\begin{frame}
\frametitle{Kinetic Theory and the Physics of MCT}
\textbf{First-Order Boltzmann Stress}
\begin{align*}
\begin{split}
 \label{surfaceBoltzmann}
 t^{\text{Bol},1}_{\alpha \beta} &= -\rho\int 
v_\alpha'v_\beta' \ g \ d^3v' \ d^3 \omega'   \\
&= \bm{ n\tau\theta(\frac{\partial U_\alpha}{\partial x_\beta} 
+ \frac{\partial U_\beta}{\partial x_\alpha}) - 
\frac{n\tau\theta}{3}(\frac{\partial U_l}{\partial x_l} \delta_{\alpha \beta}) }
\end{split}
\end{align*}
\normalsize
\pause
\textbf{Note 1}: Form matches classical dissipative stress:
\begin{equation*}
 t^{d}_{\alpha \beta} = \mu (\frac{\partial U_{\alpha}}{\partial 
x_{\beta}} + \frac{\partial U_{\beta}}{\partial x_{\alpha}}) + \lambda 
\frac{\partial U_{l}}{\partial x_{l}}\delta_{\alpha \beta}
\end{equation*}
\pause
\textbf{Note 2}: Zeroth order Boltzmann stress previously supplied a term 
related to pressure via \textbf{ideal gas law}
\begin{equation*}
 t^{Bol, 0}_{\alpha \beta} = n\theta \delta_{\alpha \beta}
\end{equation*}

\end{frame}

\begin{frame}
\frametitle{Kinetic Theory and the Physics of MCT}
Evaluation of Stress and Heat Flux Tensors: \textbf{First-Order Curtiss Stress}
\begin{align*}
\begin{split}
 \label{surfaceCurtiss}
 t^{\text{Cur},1}_{\alpha \beta} &= -\rho\epsilon_{\beta 
pq}r_{p}\int \omega_q'v_{\alpha}' \
g \ d^3v' \ d^3 \omega' \\
 &= \bm{(n\tau\theta) \epsilon_{\beta pq}r_q \frac{\partial W_p}{\partial 
x_\alpha}} 
\end{split}
\end{align*}
\pause
\textbf{Note 1}: No classical counterpart for this stress $\rightarrow$ Unique 
because of effect of particle rotation \\
\vspace{5mm}
\pause
\textbf{Note 2}: Adds effect of rotation to overall shear stress
\end{frame}

\begin{frame}
\frametitle{Kinetic Theory and the Physics of MCT}
Evaluation of Stress and Heat Flux Tensors: \textbf{First-Order Moment Stress}
\small
\begin{align*}
\begin{split}
 \label{surfacemoment}
 m^1_{\alpha \beta} &= -\frac{3\rho j}{2} \int 
\omega_{\beta}'v_{\alpha}' \ g \ d^3v' \ d^3 \omega' 
\\ &= 
\bm{(\frac{3n\tau j\theta}{2})\frac{\partial W_{\beta}}{\partial 
x_\alpha}}
\end{split}
\end{align*}
\\
\pause
\textbf{Note 1}: Also no classical analogue to this stress \\
\vspace{5mm}
\pause
\textbf{Note 2}: Gyration gradient is found in moment stress in MCT, in the 
form of $$\gamma b_{\beta \alpha} = \gamma \frac{\partial W_{\beta}}{\partial 
x_{\alpha}}$$\\
\vspace{5mm}
\pause
\textbf{Note 3}: Property $\langle v'_{\alpha} v'_{\beta} \rangle = \langle 
v'^2 \rangle \frac{\delta_{\alpha \beta}}{3}$ is employed
\end{frame}

\begin{frame}
 \frametitle{Kinetic Theory and the Physics of MCT}
First-Order Stresses + Body Forces $\rightarrow$ Boltz. Eq.
\pause
\footnotesize
 \begin{align*}
%\label{governingContsimple}
&\textbf{Continuity} \nonumber \\ 
&\frac{\partial }{\partial t}\rho + \frac{\partial }{\partial x_i}(\rho U_i) = 
0 \\
%\label{governingLinMomentsimple}
&\textbf{Linear Momentum} \nonumber \\ 
&\frac{\partial}{\partial t}(\rho U_j)  + \frac{\partial}{\partial x_i}(\rho 
U_iU_j) + \frac{\partial P}{\partial x_j} - n\tau\theta (\frac{\partial^2 
U_j}{\partial x_i \partial x_i}) - \frac{2}{3}n\tau\theta(\frac{\partial^2 
U_i}{\partial x_i \partial x_j}) - 
\\ 
& n\tau\theta\epsilon_{jkm}\frac{\partial W_m}{\partial x_k} - F_{j}= 0 
\nonumber \\
%\label{governingAngMomentsimple}
&\textbf{Angular Momentum} \nonumber \\ 
&\frac{\partial}{\partial t}(\rho j W_j) + 
\frac{\partial}{\partial x_{i}}(\rho jW_jU_i) - 
n\tau j\theta \frac{\partial W_j}{\partial x_i \partial x_i} - \bm{L_{j}} = 
0 
\nonumber\\
%\label{governingEnergysimple}
&\textbf{Energy} \\
&\frac{\partial}{\partial t}(\rho e) + \frac{\partial}{\partial 
x_{l}}(\rho eU_l)   - (4n\tau\theta)\frac{\partial^2 \theta}{\partial 
x_{l}\partial x_l} - \rho \langle v_{l}\frac{\partial e}{\partial x_{l}}
\rangle - \rho H = 0 
% \label{governingEnergysimple}
% &\textbf{Energy} \nonumber \\
% &\frac{\partial}{\partial t}(\rho e) + \frac{\partial}{\partial 
% x_{i}}(\rho eU_i)   - (4n\tau\theta)\frac{\partial^2 \theta}{\partial 
% x_{i}\partial x_i} 
%  - \rho <v_{i}\frac{\partial e}{\partial x_{i}}> = 0 
\end{align*}
\pause
Additional body force $\bm{L_{j}}$ can arise from interaction with other 
particles
\end{frame}

\begin{frame}
\frametitle{Kinetic Theory and the Physics of MCT}
 \textbf{\large Angular Momentum Body Force}
 \begin{multicols}{2}
  \begin{figure}
  \centering
   \includegraphics[width=65mm]{RelRotation_Polyatomic.png}
%    \caption{COMSOL, 2017}
  \end{figure}
  \centering 
  %(COMSOL, 2017)\\
  \Large
  \begin{itemize}
  \item Torque experienced due to coupling of gyration and angular velocity\\
  \item Strength of the coupling force due to $\nu$ \\
  \end{itemize}
   $$ L_{j} = \nu(\epsilon_{jik}U_{k,i} - 2W_{j})$$
 \end{multicols}

\end{frame}

\begin{frame}
 \frametitle{Kinetic Theory and the Physics of MCT}
 \textbf{MCT}
 \begin{align*}
\frac{\partial}{\partial t}(\rho U_j)  &+ \frac{\partial}{\partial x_i}(\rho 
U_iU_j) + \frac{\partial P}{\partial x_j} \\ &- {\color{blue}(\lambda + 
\mu)\frac{\partial^2 U_i}{\partial x_i \partial x_j}} - 
{\color{ao(english)}(\mu + 
\kappa)\frac{\partial^2 U_j}{\partial x_i\partial x_i}} - {\color{red}\kappa 
\epsilon_{jlm}\frac{\partial W_m}{\partial x_l}} - F_{j} = 0
\end{align*}
\textbf{Advanced Kinetic Theory}
\begin{align*}
 %\label{governingLinMomentrestate2}
\frac{\partial}{\partial t}(\rho U_j)  &+ \frac{\partial}{\partial x_i}(\rho 
U_iU_j) + \frac{\partial P}{\partial x_j} \\ 
&- {\color{blue}\frac{2}{3}n\tau\theta\frac{\partial^2 U_i}{\partial x_i 
\partial x_j}} - {\color{ao(english)}n\tau\theta \frac{\partial^2 U_j}{\partial 
x_i \partial x_i}} -  {\color{red}n\tau\theta\epsilon_{jlm}\frac{\partial 
W_m}{\partial x_l}} - F_{j} = 0
\end{align*}
\pause
\textbf{Note}: Equal weighting of gyration gradient and velocity Laplacian in 
Kinetic Approach, \textbf{not} in MCT
\end{frame}

\begin{frame}
\frametitle{Kinetic Theory and the Physics of MCT}
 \textbf{Navier-Stokes}
\begin{align*}
\frac{\partial}{\partial t}(\rho U_j)  &+ \frac{\partial}{\partial x_i}(\rho 
U_iU_j) + \frac{\partial P}{\partial x_j} \\ &- (\lambda + 
\mu)\frac{\partial^2 U_i}{\partial x_i \partial x_j} - \mu\frac{\partial^2 
U_j}{\partial x_i\partial x_i} - \rho F_{j} = 0
\end{align*}
\textbf{Advanced Kinetic Theory}
\begin{align*}
% \label{governingLinMomentrestate}
\frac{\partial}{\partial t}(\rho U_j)  &+ \frac{\partial}{\partial x_i}(\rho 
U_iU_j) + \frac{\partial P}{\partial x_j} \\ 
&- \frac{2}{3}n\tau\theta\frac{\partial^2 U_i}{\partial x_i 
\partial x_j} - n\tau\theta \frac{\partial^2 
U_j}{\partial x_i \partial x_i} -  
{\color{red}n\tau\theta\epsilon_{jlm}\frac{\partial W_m}{\partial x_l}} - 
\rho F_{j} 
= 0
\end{align*}
\textbf{Note}: Like MCT, Kinetic Approach adds single term from gyration to 
linear momentum equation
\end{frame}

\begin{frame}
 \frametitle{Kinetic Theory and the Physics of MCT}
 \textbf{Type I N-S Equation}: Derived using Boltzmann Distribution Function 
(Huang, 2017)
  \begin{align*}
\begin{split}
&\frac{\partial}{\partial t}(\rho U_s)  + \frac{\partial}{\partial x_l}(\rho 
U_sU_l) + \frac{\partial P}{\partial x_s} - 
{\color{blue} n\tau\theta\frac{\partial^2 
U_s}{\partial x_l \partial x_l}} - 
{\color{ao(english)} \frac{n\tau\theta}{3}\frac{\partial^2 U_l}{\partial x_l 
x_s} } - \rho F_{s} = 0
\end{split}
\end{align*}
\textbf{Reduced Kinetic Theory}: Setting $W_m = 0$
\begin{equation*}
% \label{governingLinMomentrestate}
\frac{\partial}{\partial t}(\rho U_s) + \frac{\partial}{\partial x_l}(\rho 
U_sU_l) + \frac{\partial P}{\partial x_s} - {\color{blue} n\tau\theta 
\frac{\partial^2 U_s}{\partial x_l \partial x_l} } - {\color{ao(english)} 
 \frac{2}{3}n\tau\theta\frac{\partial^2 U_l}{\partial x_l 
\partial x_s}} - \rho F_{s} = 0
\end{equation*}
\pause
\textbf{Form} of equations matches N-S but a departure in the coefficients is 
observed \pause $\rightarrow$ System still contains $\omega'_{i}$, which 
affects stresses 
\end{frame}

\begin{frame}
 \frametitle{Kinetic Theory and the Physics of MCT}
 \textbf{Type I N-S Equation}
 \begin{align*}
\begin{split}
&\frac{\partial}{\partial t}(\rho U_s)  + \frac{\partial}{\partial x_l}(\rho 
U_sU_l) + \frac{\partial P}{\partial x_s} - 
{\color{blue} n\tau\theta\frac{\partial^2 
U_s}{\partial x_l \partial x_l}} - 
{\color{ao(english)} \frac{n\tau\theta}{3}\frac{\partial^2 U_l}{\partial x_l 
x_s} } - \rho F_{s} = 0
\end{split}
\end{align*}
\textbf{Type II N-S Equation}: From B-C Distribution and {\color{red}$W_{m} = 
\frac{1}{2}\epsilon_{mab}\frac{\partial U_{b}}{\partial x_a}$}
\begin{align*}
&\frac{\partial}{\partial t}(\rho U_s)  + \frac{\partial}{\partial x_l}(\rho 
U_sU_l) + \frac{\partial P}{\partial x_s} - 
{\color{blue} \frac{n\tau \theta}{2}\frac{\partial^2 U_s}{\partial x_l \partial 
x_l}} - 
{\color{ao(english)} \frac{7n\tau \theta}{6}\frac{\partial^2 U_{p}}{\partial 
x_s \partial x_p}} - 
\rho F_s = 0
\end{align*}
\pause
\textbf{Observation}: Contribution of particle rotation still present even when 
local and macroscopic rotation are equivalent \pause $\rightarrow$ Presence of 
mean gyration, $W_m$, affects final form of expressions for coefficients 
\end{frame}


% \begin{frame}
%  \frametitle{Turbulence Modeled by MCT}
%  \begin{figure}
%   \includegraphics[width=.77\linewidth]{Slide40.png}
%  \end{figure}
% 
% \end{frame}

% \begin{frame}
% \frametitle{Calculation of Parameters}
% \Large
% Expressions for viscous diffusion, rotational Cauchy stress depend on number 
% density $n$, equilibrium temperature $\theta$, and relaxation time $\tau$\\
% \medskip
% $n$, $\theta$ can be measured in the fluid relatively easily\\
% \medskip 
% \pause
% $\rightarrow$ \textbf{Relaxation time}, $\tau$, is more difficult to assess due 
% to the need to achieve translational and gyrational equilibrium
% \end{frame}

% \begin{frame}
%  \frametitle{Kinetic Theory and the Physics of MCT}
%  \textbf{Relaxation Time}\\
%  \begin{itemize}
%   \item Additional degrees of freedom from gyration complicate discussion of 
% physics behind $\tau$
% \pause
% \item For isotropic fluids, asymmetry in stress tensor proportional to 
% $\epsilon_{jik}v_{k,i} - 2\omega_{j}$ (De Groot and Mazur, 1962) 
% $\rightarrow$ \textbf{Departure from equilibrium}
% \pause
% \item Characteristic time in decay of gyration to angular velocity found in 
% considering motion of fluid as a rigid body (De Groot and Mazur, 1962) 
% $$\frac{d\omega_j}{dt} = -\frac{\kappa}{\rho j}(2\omega_j - 
% \epsilon_{jik}v_{k,i}) ; \ \  \omega_{j} = \frac{1}{2}\epsilon_{jik}v_{k,i} (1 
% - e^{-\frac{t}{\tau}})$$
% % \item In short time scales, when departure of $\omega_{i}$ can be resolved, 
% % relaxation of internal rotation can be tracked
% \pause
% \item Relaxation time for this motion has the expression (De Groot and Mazur, 
% 1962; Evans and Streett, 1978) $$\tau = \frac{\rho j}{2\kappa}$$
% % \item Numerical simulations of MCT where fluid moves like a rigid body can show 
% % whether this is a good approximation for $\tau$
%  \end{itemize}
% \end{frame}

\begin{frame}
 \frametitle{Kinetic Theory and the Physics of MCT}
 \begin{figure}
  \includegraphics[width=.9\linewidth]{KineticMap.jpg}
 \end{figure}

\end{frame}

\begin{frame}
 \frametitle{Assessing the Equations: Numerical Results of MCT}
 \textbf{Incompressible Flow over Flat Plate}: \\
 Roughness Boundary Condition Imposed: $\omega = -\nabla \times \textbf{v}$
 \begin{figure}
  \includegraphics[width=\linewidth]{FlatPlateSetup.png}
 \end{figure}
\end{frame}
\begin{frame}
 \frametitle{Assessing the Equations: Numerical Results of MCT}
 \small
 Incompressible simulations capture velocity profiles from ERCOFTAC for 
leading-edge boundary layer $Re = \frac{\rho U_{\infty}L_{plate}}{\mu + 
\kappa} = 10^6$ with $\rho = 1$ normalized. \\ (ERCOFTAC, 1990).
 \begin{multicols}{2}
 \centering
 \textbf{Transitional}
  \begin{figure}
   \includegraphics[width = .9\linewidth]{transition.jpg}
  \end{figure}
  \centering
  \textbf{Turbulent}
\begin{figure}
 \includegraphics[width = .9\linewidth]{turbulent.jpg}
 \end{figure}
 \end{multicols}
\end{frame}

\begin{frame}
 \frametitle{Assessing the Equations: Numerical Results of MCT}
 \textbf{Dimensionless Parameters from MCT}\\
 \Large
 \begin{itemize}
 \item Peddieson found dimensionless parameters that led to aspects of 
turbulence (Peddieson, 1972)
 \pause
 \item These parameters follow from non-dimensional forms of MCT (Wonnell, 2017)
 \pause
 \item Key parameter involves coupling coefficient and traditional viscosity
$$\frac{\kappa}{\mu}$$ 
 \pause
 \item Ratio of contribution to Cauchy stress from gyration gradient to 
classical viscous diffusion 
\end{itemize}
\end{frame}
\begin{frame}
\frametitle{Assessing the Equations: Numerical Results of MCT}
 \begin{table}[h!]
\centering
 \begin{tabular}{||c c c||} 
 \hline
 Parameter & Transitional & Turbulent \\ [0.7ex] 
 \hline\hline
 $ \kappa$ & $9.6 \times 10^{-6}$ & $9.9 \times 10^{-6}$ \\ 
 $ \gamma$ & $2.2 \times 10^{-13}$ & $4.7 \times 10^{-14}$ \\
 ${\color{red} \frac{\kappa}{\mu}}$ & {\color{red} 24 } & {\color{red} 99} \\
 $\frac{\kappa}{\rho\sqrt{j}U_{\infty}}$ & 0.00135 & 0.0014 \\
 ${\color{blue} \frac{\gamma}{\mu j}}$  & 0.275 & 0.235 \\ [1ex] 
 \hline
 \end{tabular}
\label{table:material}
\end{table}
\pause
\Large
Physics of transition and turbulence driven by $\kappa >> \mu$ 
\pause $\rightarrow$ \textbf{Consistent with Kinetic Approach}\\
\pause
$\gamma$ is weighted by the traditional viscosity, $\mu$, and microinertia, $j$ 
\pause $\rightarrow$ \textbf{Also consistent with Kinetic Approach}
\end{frame}

\begin{frame}
\frametitle{Assessing the Equations: Numerical Results of MCT}
Turbulent Fluctuations Flow past a Cylinder with Turb. KE Spectrum, i.e. 
$\frac{1}{2}\rho j \omega^2$: $$E(k) \simeq (\frac{k}{k_{o}})^4 u_{o}^2 
e^{-2(\frac{k}{k_{o}})^2}$$
\begin{multicols}{2}
\begin{table}[h!]
\centering
 \begin{tabular}{||c | c||} 
 \hline
 Parameter & Value \\ [2ex] 
 \hline\hline
 ${\color{red}\alpha_{1} = \frac{\kappa}{\mu}}$ & 99 \\ 
 $\alpha_{2} = \frac{\kappa}{\rho \sqrt{j}U_{\infty}}$ & 0.0014 \\ 
 $\alpha_{3} = \frac{\gamma}{\mu j}$ & 0.235 \\ 
 $M_{t} = \frac{u_{o}}{c}$      & 0.14\\ 
 $Re = \frac{\rho_{\infty} u_{o}}{\mu k_{o}}$ & 16.7 \\ 
 $M_{\infty}$ & 2\\ 
 $L$ & 10 m\\ 
 \hline
 \end{tabular}
\end{table}
\begin{figure}
\includegraphics[width = \linewidth]{Cylinder_48k_drawing.jpg}
\end{figure}
\end{multicols}
\end{frame}

\begin{frame}
 \frametitle{Assessing the Equations: Numerical Results of MCT}
 Mach 2 Flow Past a Cylinder $\rightarrow$ Same Material Parameters as 
Incomp. Flow \pause $\rightarrow$ Relation $\kappa >> \mu$, \textbf{also found 
by Kinetic Approach}, gave meaningful turb. compressible data \pause
 \begin{multicols}{2}
\begin{figure}
 \centering
  \includegraphics[width = 
.8\linewidth]{GyrationSpectrumComparisonArrow.jpg}
\end{figure}
\begin{figure}
  \includegraphics[width = 
.8\linewidth]{TranslationMagnitudeComparisonArrow.jpg}
\end{figure}
\end{multicols}
\small
Small wavelength eddies lose energy in interaction with shock wave and a 
transfer of energy is observed to translational kinetic energy (TKE) for low 
wavenumbers \pause $\rightarrow$ \textbf{inverse energy cascade}. 
\end{frame}

\begin{frame}
 \frametitle{Assessing the Equations: Numerical Results of MCT}
 Transonic Flow over an Axisymmetric Hill
  \begin{figure}
   \includegraphics[width=.8\linewidth]{InternalMesh.png}
  \end{figure}
 \end{frame}
\begin{frame}
 \frametitle{Assessing the Equations: Numerical Results of MCT}
 Inlet Turbulent BL Profiles Set to DNS Data (Castagna, 2012), but matched 
closely with experimental data (Simpson, 2002)
 \begin{multicols}{2}
  \begin{figure}
   \includegraphics[width=\linewidth]{Inlet_Profile_Simpson_DNS.jpg}
  \end{figure}
\begin{figure}
 \includegraphics[width=\linewidth]{RMS_Velocity_Plots.jpg}
\end{figure}
 \end{multicols}
\end{frame}
 \begin{frame}
 \frametitle{Assessing the Equations: Numerical Results of MCT}
 \begin{table}[h!]
 \begin{tabular}{||c | c||} 
 \hline 
 Parameter & Value \\  [.5ex]
 \hline
 ${\color{red}\alpha_{1} = \frac{\kappa}{\mu}}$ & 99 \\
 $\alpha_{2} = \frac{\kappa}{\rho \sqrt{j}U_{\infty}}$  & 0.0014 \\
 $\alpha_{3} = \frac{\gamma}{\mu j}$ & 0.235 \\  
 $\delta_{inlet}$ & 0.039 m \\
 $H_{bump}$ & 0.078 m \\
 $M_{\infty} = \frac{U_{\infty}}{c} $      & 0.6\\
 $Re = \frac{\rho_{\infty} U_{\infty}H}{\mu + \kappa}$ & 6709\\
 $N_{cells}$ & $\mathbf{4.5509 \times 10^6}$ \\ \hline
 \end{tabular}
\end{table}
\pause
$\rightarrow$ \textbf{Castagna's mesh}: $5.4 \times 
10^7$ cells (Castagna, 2014)\\
\vspace{2mm}
\pause
Turbulence parameters continually rely on $\kappa >> \mu$, consistent with 
findings of \textbf{Kinetic Theory}\\
\end{frame}

\begin{frame}
\frametitle{Assessing the Equations: Numerical Results of MCT}
Surface Pressure Data\pause $\rightarrow$ \textit{Validation} of turbulent MCT 
simulation with experimental data
\begin{figure}
 \centering
  \includegraphics[width=.5\linewidth]{Transonic_Bump_C_p_Comparison.jpg}
\end{figure}
\pause
Turbulent Boundary Layer led to better surface pressure data, and separation of 
the BL on the windward side of the bump
\end{frame}

\begin{frame}
 \frametitle{Assessing the Equations: Numerical Results of MCT}
 \Large
\textbf{Inflow turbulence} with $\kappa >> \mu$ $\rightarrow$ Periodic Vortices
\animategraphics[autoplay,loop,width=\linewidth]{12}
{Vortices_Animation.}{0478}{0498}
%\mediabutton[jsaction={anim.myAnim.playFwd();}]{\fbox{\strut Play}}
%\mediabutton[jsaction={anim.myAnim.pause();}]{\fbox{\strut Pause}}
\end{frame}

\begin{frame}
 \frametitle{Discussion}
 \large
 \begin{itemize}
  \item Kinetic theory takes \textit{discrete} approach to fluid flows \pause
$\rightarrow$ Any framework is an approximation of a continuous body
\pause
\item Understanding flows with local rotation from discrete approach gives a 
deeper physical basis for resulting equations and expressions
\pause
\item Presumed single \textbf{relaxation time} and single \textbf{equilibrium 
temperature} for all motions
\pause
\item Actual physical particles will not be perfect spheres \pause 
$\rightarrow$ Model should improve as particles approach spherical shape
\pause
\pause
\item Relaxation time requires numerical and experimental data to validate 
 \end{itemize}

\end{frame}


\begin{frame}
\frametitle{Conclusion and Future Work}
\begin{itemize}
 \item First-order approximation to Boltzmann-Curtiss equations yields 
equations that match the form of the MCT equations
\pause
\item Cauchy stress and viscous diffusion weighted equally by coefficient 
$n\tau\theta$
\pause
\item MCT data suggest $\kappa >> \mu$ and $\gamma \sim \mu j$, confirming 
findings by Kinetic Theory
\pause
\item Meaningful data from incompressible profiles, and validation with exp. 
data in MCT with $\kappa >> \mu$
\pause
\item Direct simulations of kinetic theory first step to testing their 
effectiveness
\pause
\item Simulations of hypersonic and turbulent flows in MCT and kinetic 
description $\rightarrow$ Validate kinetic theory expressions for MCT stresses 
and coefficients
\end{itemize}
\end{frame}



% \begin{frame}
% \textnormal
% Presumes fluid to be a \textbf{morphing continuum}: A set of finite size 
% structures that possess their own independent subscale motion, $\xi$, 
% apart from their macromotion, $x$. The subscale motion possesses its own 
% orientation, $\chi$, independently from its macromotion orientation, $k$. 
% \frametitle{Morphing Continuum Theory}
% \begin{multicols}{2}
% \begin{figure}
% \includegraphics[width=\linewidth]{mctdirectors.png}
% \end{figure}
% \large
% \textbf{Kinematics}
% \begin{equation*} %\label{direc1}
% %\begin{split}
% \centering
% {\it x_{k} = x_{k} (X_{K},t)}
% \quad
% {\it  X_{K} = X_{K} (x_{k},t)}
% \end{equation*}
% \begin{equation*} %\label{direc2}
% \centering
% {\it \xi_{k} = \chi_{kK}(X_{K},t)\Xi_{K}}
% \quad
% {\it \Xi_{K} =  \bar{\chi}_{Kk}\xi_{k}}
% \end{equation*}
% \begin{equation*} %\label{direc3}
% \centering
% {\it K = 1, 2, 3} 
% \quad
% {\it  k = 1, 2, 3 }
% \end{equation*}
% \textbf{Deformation Rates}
% \begin{equation*}
%  a_{kl}  = v_{l,k} + \epsilon_{lkm}\omega_{m}
% \end{equation*}
% \begin{equation*}
%  b_{kl} = \omega_{k,l}
% \end{equation*}
% 
% \end{multicols}
% \end{frame}
% \begin{frame}
% \frametitle{MCT: Balance Laws}
% \centering
% \begin{multicols}{2}
% %\begin{equation}
% \textbf{Continuity}
% $$\frac{\partial \rho}{\partial t} + (\rho v_{i})_{,i} = 0$$\\
% %\end{equation}
% %\begin{equation}
% \textbf{Linear Momentum}
% $$t_{lk,l} + \rho f_{k} =  \rho {\dot{v}}_{ k} $$\\
% %\end{equation}
% %\begin{equation}
% %\end{equation} 
% \textbf{Angular Momentum}
% $$ m_{lk,l} + \epsilon_{kij}t_{ij} + \rho l_{k} = \rho j\dot{\omega}_{ k}$$\\
% %\begin{equation}
% \textbf{Energy}
% $$\rho \dot{e} - t_{kl}a_{kl} - m_{kl}b_{lk} + q_{k,k} - \rho h = 0$$\\
% \end{multicols}
% \begin{multicols}{2}
% \begin{itemize}
% \item $t_{lk}$: Cauchy stress 
% \item $m_{lk}$: Moment stress 
% \item $j$: Microinertia of eddies
% \item $a_{bkl}, b_{kl}$: Defomation Rates
% \item $q_{k}$: Heat flux
% \item $h$: Energy source density
% \item $e$: Internal energy density
% \end{itemize}
% \end{multicols}
% %\end{equation} 
% \end{frame}
% \begin{frame}
% \frametitle{MCT: Constitutive Equations}
% \small
% Stresses, heat flux now depend upon deformation of eddies as well. 
% Constitutive equations derived from thermodynamics for isotropic 
% fluids.
% \begin{multicols}{2}
% \centering
% \textbf{\large Cauchy Stress}
% \normalsize
% $$t_{kl} = -P\delta_{kl} + \lambda tr(a_{mn})\delta_{kl} + (\mu + {\color{blue}\kappa})a_{kl} + \mu a_{lk} $$\\
% \textbf{\large Moment Stress}
% \normalsize
% {\color{violet}$$m_{kl} = \frac{\alpha_{T}}{\theta} \epsilon_{klm} \theta_{,m} + \alpha tr(b_{mn})\delta_{kl} + \beta b_{kl} + \gamma b_{lk} $$}\\
% \textbf{\large Heat Flux}
% \normalsize
% $$q_{k} = \frac{K}{\theta}\theta_{,k} + 
% {\color{red}\alpha_{T}\epsilon_{klm}\omega_{m,l} } $$\\
% \textbf{\large Deformation Rates}
% $$ a_{kl}  = v_{l,k} + {\color{magenta}\epsilon_{lkm}\omega_{m}}$$
% {\color{violet}$$ b_{kl} = \omega_{k,l} $$}
% \end{multicols}
% \end{frame}
% \begin{frame}
% \frametitle{MCT: Governing Equations}
% \small
% Now have independent equation governing \textbf{gyration} of 
% eddies. Coupling coefficient $\color{blue} \kappa$ ties eddy rotation to linear 
% momentum equation. 
% \vfill
% \centering
% \textbf{Linear Momentum}
% \normalsize
% $$-P_{,k} + (\lambda + \mu)v_{l,lk} + (\mu + {\color{blue}\kappa})v_{k,ll} + {\color{blue}\kappa\epsilon_{klm}\omega_{l,m}} = \rho\dot{v}_{k}$$ \\
% \textbf{\color{violet}Angular Momentum}
% \normalsize
% $$\it (\alpha + \beta)\omega_{l,lk} + \gamma \omega_{k,ll} + 
% {\color{blue}\kappa(\overbrace{\epsilon_{klm}v_{l,m} - 
% 2\omega_{k})}^{{\color{black}Absolute \ Rotation}}} = \rho j\dot{\omega}_{k}  
% $$\\
% \pause
% \textbf{Energy}
% $$({\color{blue}t_{kl}}v_{l})_{,k} + {\color{violet}(m_{kl}\omega_{l})_{,k}} - q_{k,k} + \rho h  = \rho{\color{cyan}\dot{E}} $$ \\
% \textnormal
% where $E = e + \frac{1}{2}(v_{l}v_{l} + {\color{cyan}j\omega_{l}\omega_{l}})$ 
% now contains turbulent kinetic energy. 
% \end{frame}
% \begin{frame}
%  \frametitle{MCT: Success with Incompressible Turbulence}
%  \large
%  Simulations capture velocity profiles from ERCOFTAC for leading-edge 
% boundary layer $Re = 10^6$ (ERCOFTAC, 1990).
%  \begin{multicols}{2}
%  \centering
%  \textbf{Transitional}
%   \begin{figure}
%    \includegraphics[width = \linewidth]{transition.jpg}
%   \end{figure}
%   \centering
%   \textbf{Turbulent}
% \begin{figure}
%  \includegraphics[width = \linewidth]{turbulent.jpg}
% \end{figure}
% 
%  \end{multicols}
% 
% \end{frame}
% \begin{frame}
% \frametitle{Numerical Setup}
% Dimensionless parameters chosen according to values used successfully for 
% incompressible turbulence. Balance of eddy contribution to Cauchy stress with 
% viscous diffusion, captured by $\color{red} \alpha_{1}$, critical for 
% generation of turbulent flow.
% \begin{multicols}{2}
% \begin{table}[h!]
% \centering
%  \begin{tabular}{||c | c||} 
%  \hline
%  Parameter & Value \\ [2ex] 
%  \hline\hline
%  ${\color{red}\alpha_{1} = \frac{\kappa}{\mu}}$ & 99 \\ 
%  $\alpha_{2} = \frac{\kappa}{\rho \sqrt{j}U_{\infty}}$ & .0014 \\ 
%  $\alpha_{3} = \frac{\gamma}{\mu j}$ & .235 \\ 
%  $M_{t} = \frac{u_{o}}{c}$      & .14\\ 
%  $Re = \frac{\rho_{\infty} u_{o}}{\mu k_{o}}$ & 16.7 \\ 
%  $M_{\infty}$ & 2\\ 
%  $L$ & 10 m\\ 
%  \hline
%  \end{tabular}
% \end{table}
% \begin{figure}
% \includegraphics[width = \linewidth]{Cylinder_48k_drawing.jpg}
% \end{figure}
% \end{multicols}
% \end{frame}
% 
% \begin{frame}
% \frametitle{Energy Spectrum}
% \large
% \begin{itemize}
% \item Monitor turbulent kinetic energy, embodied by the term $\frac{1}{2}\rho 
% j(\omega_{z})^2$
% \item Implement gyration $\omega_{z}$ at inlet as a sine wave of form $Asin(\nu t)$
% \item Set magnitude of gyration to $\frac{u_{o}}{\sqrt{j}}$ where $u_{o}$ is 
% the RMS velocity set by $M_{t}$
% \item Choose spectrum used by Lee et al to model low-Re ($\frac{\rho_{\infty} 
% u_{o}}{\mu k_{o}}$) turbulence (Lee, 1997)
% \end{itemize}
% $$E(k) \simeq (\frac{k}{k_{o}})^4 u_{o}^2 e^{-2(\frac{k}{k_{o}})^2}$$\\
% \begin{itemize}
% \item Here $k_{o} = 4$ is the maximum energy wavenumber
% \item $k = \frac{\nu}{U_{\infty}}$ is generated by random number generator
% \end{itemize}
% \end{frame}
% \begin{frame}
% \frametitle{Results: Energy Spectra}
% \begin{multicols}{2}
% \begin{figure}
%  \centering
%   \includegraphics[width = 
% .8\linewidth]{GyrationSpectrumComparisonArrow.jpg}
% \end{figure}
% \begin{figure}
%   \includegraphics[width = 
% .8\linewidth]{TranslationMagnitudeComparisonArrow.jpg}
% \end{figure}
% \end{multicols}
% Small wavelength eddies lose energy in interaction with shock wave and a 
% transfer of energy is observed to translational kinetic energy (TKE) for low 
% wavenumbers \pause $\rightarrow$ \textbf{inverse energy cascade}.
% \end{frame}
% \begin{frame}
%  \frametitle{Eddy Amplification}
%  \begin{multicols}{2}
%  \begin{figure}
%   \includegraphics[width = \linewidth]{PressureContour.jpg}  
%  \end{figure}
%  \begin{figure}
%   \includegraphics[width = \linewidth]{AbsoluteRotationAmplify.jpg}
%  \end{figure}
% \end{multicols}
% \begin{multicols}{2}
% \begin{figure}
%  \includegraphics[width = .8\linewidth]{PressureXcoordinate.jpg}
% \end{figure} 
% Sharp increases in pressure result in development of eddy structure and 
% geometry. Size and shape of individual eddies after the shock can be discerned 
% through the absolute rotation ($\epsilon_{klm}v_{l,m} - 
% 2\omega_{k}$).
% \end{multicols}
% \end{frame}
% \begin{frame}
% \frametitle{Eddy Dissipation}
% \begin{multicols}{2}
% \begin{figure}
% \centering
% \includegraphics[width = \linewidth]{AbsoluteRotationEddyDissipation.jpg}
% \end{figure}
% \begin{figure}
% \centering
%  \includegraphics[width= \linewidth]{LogPressureContourEddyDissipation.jpg}
% \end{figure}
% \end{multicols}
% Logarithmic scale used for pressure contours. Sharp drops in pressure show 
% lines of eddy dissipation. Angle of eddy disspation of approximately 14 
% degrees formed after the shock.
% \end{frame}
% \begin{frame}
% \frametitle{Conclusion}
% \large
% \begin{itemize}
% \item MCT introduced to capture multiscale interactions between eddies and 
% shock wave
% \pause
% \item Spectra show interaction for wide range of eddy sizes
% \pause
% \item Spike in post-shock TKE means eddies transferred energy \pause
% $\rightarrow$ \textbf{inverse energy cascade}
% \pause
% \item Eddy rotation noticeably amplified after shock \pause $\rightarrow$ N-S 
% equations break down after shock
% \pause
% \item Eddy dissipation lines form when sharp declines in pressure exist
% \end{itemize}
% \end{frame}
\begin{frame}
 \frametitle{Publications}
 \textbf{Articles}
 \tiny
\begin{enumerate}
 \item Louis B. Wonnell, and James Chen. "A First-Order Approximation to the
Boltzmann-Curtiss Equations for Flows with Local Spin," Under Review
\item Louis B. Wonnell, Mohamad I. Cheikh and James Chen. ``A Morphing 
Continuum 
Simulation of Transonic Flow over an Axisymmetric Hill,'' Under Review
\item Mohamad Ibrahim Cheikh, Louis B. Wonnell, and James Chen. "Morphing 
continuum analysis of energy transfer in compressible turbulence." 
\textit{Physical Review Fluids} \textbf{3}, no. 2 (2018): 024604.
\item Louis B. Wonnell, and James Chen. "Morphing Continuum Theory: 
Incorporating the Physics of Microstructures to Capture the Transition to 
Turbulence Within a Boundary Layer." \textit{Journal of Fluids Engineering} 
\textbf{139}, no. 1 (2017): 011205. 
\item Louis B. Wonnell, and James Chen. "Extension of Morphing Continuum Theory 
to Numerical Simulations of Transonic Flow over a Bump." \textit{Proceedings of 
the 47th AIAA Fluid Dynamics Conference}, AIAA 2017-3461
\item Louis B. Wonnell, and James Chen. "A Morphing Continuum Approach to 
Compressible Flows: Shock Wave-Turbulent Boundary Layer Interaction." 
\textit{Proceedings of the 46th AIAA Fluid Dynamics Conference}, AIAA 2016-4279.
\end{enumerate}
\normalsize
\textbf{Conference Works}
\tiny
\begin{enumerate}
\item Louis B. Wonnell, Mohamad Ibrahim Cheikh, and James Chen. "Morphing 
Continuum Theory: A First Order Approximation to the Balance Laws." 70th Annual 
Meeting of the APS Division of Fluid Dynamics. Denver, CO. November 19-21, 2017.
\item Louis B. Wonnell, and James Chen. "Extension of Morphing Continuum Theory 
to Numerical Simulations of Transonic Flow over a Bump." 47th AIAA Fluid 
Dynamics Conference. Denver, CO. June 5-9, 2017.
\item Louis B. Wonnell, and James Chen. "A Morphing Continuum Approach to 
Compressible Flows: Shock Wave-Turbulent Boundary Layer Interaction." 46th AIAA 
Fluid Dynamics Conference. Washington, D.C. June 13-17, 2016.
\item James Chen, and Louis Wonnell. "A Multiscale Morphing Continuum 
Description for Turbulence." 68th Annual Meeting of the APS Division of Fluid 
Dynamics. Boston, MA. November 22-24, 2015.
\end{enumerate}


\end{frame}


\begin{frame}
\Huge{\centerline{Thank you for your attention}}
\centering
\begin{multicols}{2}
\begin{figure}
 \includegraphics[width=\linewidth]{AFOSR.png}
\end{figure}

\begin{figure}
 \includegraphics[width=\linewidth]{MCPL.jpg}
\end{figure}
\end{multicols}
\large
\centering
\textbf{James Chen}: jmchen@ksu.edu\\
\textbf{Louis Wonnell}: lwonnell@ksu.edu.
\end{frame}

%----------------------------------------------------------------------------------------

\end{document} 
